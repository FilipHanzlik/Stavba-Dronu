\documentclass[12pt]{report}
\usepackage[margin=1in]{geometry}
\usepackage{amsmath, amssymb}
\usepackage{graphicx}
\usepackage{titlesec}

\titleformat{\chapter}[display]
  {\normalfont\bfseries}{}{0pt}{\Huge}
  
\renewcommand*\contentsname{Obsah}

\begin{document}


\begin{titlepage}
	\begin{flushright}
	5.D \\
	2022 \\
	Gymnázium Jana Nerudy
	\end{flushright}
	\vspace*{6cm}
	
	\begin{center}
		\textbf{\huge Stavba a Programování Dronu} \\
		\textbf{\huge Závěrečná zpráva } \\[\baselineskip]
		{\LARGE Jan Vorel a Filip Hanzlík}
	\end{center}		
\end{titlepage}


\tableofcontents

\chapter{Zadání projektu}
\section{Tým}
Tento projekt byl realizován, jako studentský projekt, pod záštitou Gymnázia Jana Nerudy.

\section{Cíl a Motivace}
V dnešní době drony rapidně rostou na popularitě a vývoj neustále přináší nové možnosti. Drony se tak stávají velmi atraktivní oblastí pro výzkum a to i na nižší úrovni, což je reflektováno i jejich častější přítomností v médiích. Co drony tvoří tak unikátními, je nejen výborná vzdušná mobilita a ovládání, ale především možnost nadstavby sofwarem, jež dále výrazně rozšiřuje využití těchto přístrojů. Již aktuálně, díky implementaci systémů pro snímání obrazů, automatického předurčování trajektorie a vzájemné koordinace více dronů, je jejich uplatnění velmi široké, a to i v komerční sféře. Jedná se například o mapování a analyzování nedostupných teritorií, pořizování videozáznamů z těžko dostupných vzdušných úhlů, automatické doručování zásilek, aplikace v armádě a mnoho dalších využití. Jde tedy o technologii se zdánlivě velkým a rozvíjejícím se potenciálem.  \par 
Cílem projektu bude tedy, jak mu již název částečně napovídá, sestrojení letu schopného dronu, a to ze základních elektronických součástek a senzorů spolu s vytvořením softwaru pro jeho ovládání. Tento projekt tedy bude mít důraz na programovací aspekty, práce s dronem, zahrnující zejména práci s ovládáním mikrokontroléru, sběru dat ze senzorů/přijímačů, patřičného ovládání motorů a správné zpracování všech těchto dat, pro jejich finální implementaci v algoritmu, který bude na jejich základě provádět korekce v stavu dronu. Jako primární cíl toho projektu bychom tedy definovali sestrojení dronu, jež bude schopný na základě dálkového ovládání, udržet konstantní vzdušnou pozici po dobu alespoň jedné minuty. Základním předpokladem zde je, že dron, schopný tohoto úkolu, potřebuje všechny systémy, které jsou potřebné pro funkčnost letu. Tedy dron schopný se aktivně udržet na místě, musí být zároveň plně schopný letu. Takto již sestavený dron by nadále mohl být základem pro vývin dalších funkcí, které by pak už měli aplikovatelné využití. Jedná se například o instalaci kamery, z níž by mohli být zpracovávána data, která by dále mohla dále sloužit širokému spektru aplikací, jako je například vytváření 3D modelů prozkoumaných oblastí, nebo detekci objektů, či lidí. Dále, vedle samotného vývinu dronu, by tento projekt [sloužit spolužákům pro vstup do tohoto pole -> budeme se zde snažit přehledně popsat náš postup atd.]

\chapter{Plán}
\section{Rozpočet}

\chapter{Realizace}

\chapter{Závěr}


\end{document}
